
El broker astronómico ALeRCE usa un clasificador con machine learning que usa la variabilidad de la magnitud de las fuentes astronómicas, en forma de curvas de luz, para clasificar estas fuentes en un conjunto de clases como binarias eclipsantes, RR Lyrae y \texttt{active galactic nuclei}. ALeRCE se encarga también de calcular distintas características de las curvas de luz para el clasificador, y entre estas esta el periodo.

En esta memoria, se optimiza el algoritmo de cálculo de periodo, o periodograma, usado en ALeRCE, MHAOV, que calcula una medida de confianza para un conjunto de frecuencias de prueba en base a la curva de luz de un objeto. Esta optimización se logra implementando y diseñando una versión paralela en GPU del mismo algoritmo para mejorar su rendimiento, y usando un algoritmo de post proceso para mejorar su precisión. La implementación en GPU de MHAOV, denominada GMHAOV, se realizó en CUDA.

La validación de GMHAOV presenta una desviación promedio del resultado de MHAOV de casi $6.3\%$ para datos reales, pero obtiene una desviación de solo $5.58 \times 10^{-9}$ para datos generados sintéticamente. La causa de esto no se estudia a fondo, pero esta discrepancia disminuye la precisión del cálculo del periodo en solamente un $1\%$.

En la paralelización, se obtuvo un speedup máximo de GMHAOV respecto a MHAOV de $1.5$ para $7000$ frecuencias de prueba, y de $7.5$ para $700$ frecuencia de prueba. Los algoritmos de postproceso que se usaron para refinar la frecuencia, que descarta frecuencias erróneas entregadas por el periodograma y encuentra la que tiene mayor probabilidad de ser la frecuencia real, demostraron ser relevantes para una parte de las curvas de luz de binarias eclipsantes y RR Lyrae, llegando a incrementar la precisión en $10$ veces, pero siendo contraproducente para algunos casos, como con las binarias eclipsantes.

Se estudió el código y la teoría de MHAOV, junto con el código de otro periodograma implementado en GPU llamado GCE, y además se investigó como crear interfaces en Python para ejecutar código CUDA, junto con los fundamentos de la detección de señales en curvas de luz.

Como trabajo futuro se propone usar el resultado de la refinación de periodos, GCE y GMHAOV para entrenar el clasificador de ALeRCE y evaluar el impacto que tendría en su precisión. Esto permitiría decidir si vale la pena incrementar el tiempo de ejecución del cálculo de periodos para mejorar el clasificador, ejecutando GCE y los algoritmos de refinación de periodos junto con GMHAOV, ya que el speedup de GMHAOV y la eficiencia de GCE permitiría hacer esto sin un impacto mayor en el tiempo de ejecución actual del periodograma.




